\documentclass[a4paper]{scrartcl}
\usepackage{amssymb, amsmath} % needed for math
\usepackage[utf8]{inputenc} % this is needed for umlauts
\usepackage[ngerman]{babel} % this is needed for umlauts
\usepackage[T1]{fontenc}    % this is needed for correct output of umlauts in pdf
\usepackage{hyperref}   % links im text
\usepackage{parskip}
\usepackage{csquotes}

\title{Jahresbericht der HSG\\'Machine Learning Karlsruhe'}

%%%%%%%%%%%%%%%%%%%%%%%%%%%%%%%%%%%%%%%%%%%%%%%%%%%%%%%%%%%%%%%%%%%%%
% Begin document                                                    %
%%%%%%%%%%%%%%%%%%%%%%%%%%%%%%%%%%%%%%%%%%%%%%%%%%%%%%%%%%%%%%%%%%%%%
\begin{document}
\maketitle

Das Ziel von ML-KA ist es Studenten mit Interesse am maschinellen Lernen zu
Unterstützen, in selbstorganisierten Kleingruppen Projekte umzusetzen. Eine
dieser Kleingruppen ist die Paper Discussion Group (PDG) mit etwa 6 - 10
Mitgliedern pro Treffen. Als Ansprechpartner steht der Vorstand, die Facebook-Gruppe \url{https://www.facebook.com/groups/961427967221226/} mit aktuell 337~Mitgliedern sowie der E-Mail Verteiler
\verb+ml@lists.kit.edu+ zur Verfügung. Weitere Informationen teilen wir auf
unserer Website
\url{https://ml-ka.de}.

Im Kalenderjahr 2016 hat die Hochschulgruppe ML-KA folgende Aktivitäten
durchgeführt:

\begin{itemize}
    \item 13.01.2016: 7. PDG (Fully Convolutional Networks for Semantic Segmentation)
    \item 20.01.2016: 8. PDG (Understanding LSTM Networks)
    \item 27.01.2016: 3. Monatstreffen (Neural Network-based Multilingual Translation Models)
    \item 27.01.2016: 9. PDG (Recurrent Models of Visual Attention)
    \item 03.02.2016: 10. PDG (Deep Residual Learning for Image Recognition)
    \item 10.02.2016: 11. PDG (Show, Attend and Tell: Neural Image Caption Generation with Visual Attention)
    \item 17.02.2016: 12. PDG (Unsupervised Visual Representation Learning by Context Prediction)
    \item 24.02.2016: 13. PDG (Playing Atari with Deep Reinforcement Learning)
    \item 09.03.2016: Vorstandstreffen
    \item 26.04.2016: 14. PDG (Deep Networks with Stochastic Depth)
    \item 03.05.2016: 15. PDG (Faster R-CNN: Towards Real-Time Object Detection with Region Proposal Networks)
    \item 17.05.2016: 16. PDG (Understanding the difficulty of training deep feedforward neural networks)
    \item 24.05.2016: 17. PDG (Asynchronous Methods for Deep Reinforcement Learning)
    \item 31.05.2016: 18. PDG (Speech Recognition with Deep Recurrent Neural Networks)
    \item 01.06.2016: 1. GIML (Crime Prediction)
    \item 07.06.2016: 19. PDG (Distributed Representations of Words and Phrases and their Compositionality)
    \item 14.06.2016: 20. PDG (Neural Turing Machines)
    \item 15.06.2016: 2. GIML (Auswirkungen von ML auf den Arbeitsmarkt)
    \item 28.06.2016: 21. PDG (DeepFace: Closing the Gap to Human-Level Performance in Face Verification)
    \item 29.06.2016: 3. GIML (Was ist Intelligenz?)
    \item 05.07.2016: 22. PDG (Memory Networks)
    \item 12.07.2016: 23. PDG (Generating Design Suggestions under Tight Constraints with Gradient-based Probabilistic Programming)
    \item 19.10.2016: 24. PDG (Attention and Augmented Recurrent Neural Networks)
    \item 26.10.2016: 25. PDG (Generative Adversarial Nets)
    \item 02.11.2016: 26. PDG (Ausgefallen)
    \item 09.11.2016: 27. PDG (Differentiable Neural Computers)
    \item 16.11.2016: 28. PDG (Differentiable Neural Computers 2)
    \item 23.11.2016: 29. PDG (One-shot Learning with Memory-Augmented Neural Networks)
    \item 30.11.2016: 30. PDG (Learning to learn by gradient descent by gradient descent)
    \item 07.12.2016: 31. PDG
    \item 14.12.2016: 32. PDG
    \item 21.12.2016: Mitgliederversammlung
    \item 21.12.2016: 33. PDG (Tagger: Deep Unsupervised Perceptual Grouping)
\end{itemize}

Die Protokolle der verschiedenen Treffen sind unter \url{https://github.com/ML-KA/protokolle}
zugänglich. Weitere Informationen zur Paper Discussion Group (PDG) sind unter \url{https://ml-ka.de/paper-discussion-group/} und zu GIML unter \url{https://ml-ka.de/giml/}.

Außerdem hat eine Gruppe von ML-KA Mitgliedern bei einem Wettbewerb der
Herbsttagung der Arbeitsgruppe Datenanalyse und Numerische Klassifikation (AG
DANK) teilgenommen.

\vspace{2cm}
Karlsruhe, den 15. Dezember 2016\\
\vspace{1cm}\\
Martin Thoma
\end{document}
